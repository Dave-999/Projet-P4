%%%%%%%%%%%%%%%%%%%%%%%%%%%%%%%
%This is the article LaTeX template for RSC journals
%Copyright The Royal Society of Chemistry 2010
%%%%%%%%%%%%%%%%%%%%%%%%%%%%%%%


\documentclass[8.5pt,twoside,twocolumn]{article}
\oddsidemargin -1.2cm
\evensidemargin -1.2cm
\textwidth 18cm
\headheight 1.0in
\topmargin -3.5cm
\textheight 22cm
\usepackage[super,sort&compress,comma]{natbib} 
\usepackage{mhchem}
\usepackage{times,mathptmx}
% \usepackage{times}
% feel free not to use mathptmx if it causes difficulties
\usepackage{sectsty}
\usepackage{balance} 

\usepackage{graphicx} %eps figures can be used instead
\usepackage{lastpage}
\usepackage[format=plain,justification=raggedright,singlelinecheck=false,font=small,labelfont=bf,labelsep=space]{caption} 
\usepackage{fancyhdr}
\pagestyle{fancy}

\begin{document}

\thispagestyle{plain}
\fancypagestyle{plain}{
\fancyhead[L]{\includegraphics[height=20pt]{epl-logo}}
%\fancyhead[C]{\hspace{-1cm}\includegraphics[height=20pt]{headers/CH}}
\fancyhead[R]{LFSAB1505: Project 4 (in applied physics and chemistry) }
\renewcommand{\headrulewidth}{1pt}}
\renewcommand{\thefootnote}{\fnsymbol{footnote}}
\renewcommand\footnoterule{\vspace*{1pt}% 
\hrule width 3.4in height 0.4pt \vspace*{5pt}} 
\setcounter{secnumdepth}{5}



\makeatletter 
\def\subsubsection{\@startsection{subsubsection}{3}{10pt}{-1.25ex plus -1ex minus -.1ex}{0ex plus 0ex}{\normalsize\bf}} 
\def\paragraph{\@startsection{paragraph}{4}{10pt}{-1.25ex plus -1ex minus -.1ex}{0ex plus 0ex}{\normalsize\textit}} 
\renewcommand\@biblabel[1]{#1}            
\renewcommand\@makefntext[1]% 
{\noindent\makebox[0pt][r]{\@thefnmark\,}#1}
\makeatother 
\renewcommand{\figurename}{\small{Fig.}~}
\sectionfont{\large}
\subsectionfont{\normalsize} 

\fancyfoot{}
%\fancyfoot[LO,RE]{\vspace{-7pt}\includegraphics[height=9pt]{headers/LF}}
%\fancyfoot[CO]{\vspace{-7.2pt}\hspace{12.2cm}\includegraphics{headers/RF}}
%\fancyfoot[CE]{\vspace{-7.5pt}\hspace{-13.5cm}\includegraphics{headers/RF}}
\fancyfoot[RO]{\footnotesize{\sffamily{1--\pageref{LastPage} ~\textbar  \hspace{2pt}\thepage}}}
\fancyfoot[LE]{\footnotesize{\sffamily{\thepage~\textbar\hspace{3.45cm} 1--\pageref{LastPage}}}}
\fancyhead{}
\renewcommand{\headrulewidth}{1pt} 
\renewcommand{\footrulewidth}{1pt}
\setlength{\arrayrulewidth}{1pt}
\setlength{\columnsep}{6.5mm}
\setlength\bibsep{1pt}

\twocolumn[
  \begin{@twocolumnfalse}
\noindent\LARGE{\textbf{Properties characterisation of a nickel silver key}}
\vspace{0.6cm}

\noindent\large{\textbf{Florian Baulard, Robin Crits, David Dispas and Arnaud Paquet$^{\ast}$}}\vspace{0.5cm}


\noindent\textit{\small{\textbf{Major/minor in chemical and physical engineering - Université catholique de Louvain-la-Neuve, Belgium}}}
\end{@twocolumnfalse}

\begin{abstract}

\noindent \normalsize{The abstract should be a single paragraph which summarises the content of the article. Any references in the abstract should be written out in full \textit{e.g.} [Surname \textit{et al., Journal Title}, 2000, \textbf{35}, 3523].}
\vspace{0.5cm}
\end{abstract}
  ]

%Footnotes

\footnotetext{$^{\ast}$E-mail adresses: florian.baulard@student.uclouvain.be(F. Baulard), robin.crits@student.uclouvain.be(R. Crits), david.dispas@student.uclouvain.be(D. Dispas), arnaud.paquet@student.uclouvain.be(A. Paquet).}


\section{Introduction}

\section{Materials and methods}

\section{Results}

\section{Discussion}

\section{Conclusions}
\section*{Acknowledgements}






%The \balance command can be used to balance the columns on the final page if desired. It should be placed anywhere within the first column of the last page.

%\balance

%If notes are included in your references you can change the title from 'References' to 'Notes and references' using the following command:
%\renewcommand\refname{Notes and references}

\footnotesize{
\bibliography{rsc} %your .bib file
\bibliographystyle{rsc} %the RSC's .bst file
}

\end{document}
